\documentclass[twocolumn]{article}
\usepackage{graphicx}
\graphicspath{{images/}} % images path (picture folder)
\usepackage{caption}
\usepackage{subcaption}
\title{ \LaTeX}
\begin{document}
	\maketitle
	
\section{Introduction}
This is the introduction of my work

	% simple picture
	\begin{figure}
		\includegraphics{V.png}
	\end{figure}

	mentioned in Fig. \ref{experiment}

	% with parameter and specification
	\begin{figure}[!h] % make image in the (t:top, b:bottom, h:here, !:force[optional])
		\centering % make image in the center
		\caption{\label{experiment}Logo of page} % name of photo (caption)
		\includegraphics[width=0.9\textwidth]{images/Blue.png} % control width of image (change to 90% of the text width)
	\end{figure}

	\begin{figure}
	\includegraphics[width=0.9\columnwidth]{images/Red.png} % change width to 90% of the column width
	\end{figure}

\section{Related work}
This is the related work of my document, the Fig. \ref{fig:app_logo} show the logos of my app, that contains two logo (Fig. \ref{fig:blue_logo} and Fig. \ref{fig:red_logo})


	% subfigures
	\begin{figure*} % *: figure take all page with (in case we use twocolumn)
		\centering
		\begin{subfigure}[b]{0.45\columnwidth}
			\includegraphics[width=\textwidth]{Red.png}
			\caption{}
			\label{fig:red_logo}
		\end{subfigure}
		~ % ~: beside, new line (blank ligne): bottom to
		\begin{subfigure}[b]{0.45\columnwidth}
			\includegraphics[width=\textwidth]{Blue.png}
			\caption{}
			\label{fig:blue_logo}
		\end{subfigure}		
		\caption{(a) Logo Red. (b) Logo Blue.}
		\label{fig:app_logo}
	\end{figure*}
	
\end{document}